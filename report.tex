\documentclass[11pt]{article}
\usepackage{geometry}
\geometry{margin=1in}
\usepackage{enumitem}
\usepackage{hyperref}
\usepackage{graphicx}
\setlength{\parskip}{0.4em}

\title{Project Plan: Automated Dynamic Analysis Signature Generation (Project 3)}
\author{Ikram Benfellah, Fadel Fatima Zahra}
\date{January 25, 2026}

\begin{document}
\maketitle

\section{Project Information}
\begin{itemize}[leftmargin=*, itemsep=0pt]
    \item \textbf{Project Title:} Automated Dynamic Analysis Signature Generation
    \item \textbf{Team Members:} Ikram Benfellah , Fadel Fatima Zahra
    \item \textbf{GitHub Repository:} \url{https://github.com/ykram051/Automated-Dynamic-Analysis-Signature-Generation}
\end{itemize}

\section{Methodology}
\textbf{High-level Approach:}
\begin{enumerate}[leftmargin=*, itemsep=0pt]
    \item Parse CAPEv2 JSON logs to extract behavioral patterns (API calls, network, file ops).
    \item Use LLMs to identify and summarize key behaviors.
    \item Automatically generate Python-based CAPEv2 signatures.
    \item Map behaviors to MITRE ATT\&CK techniques.
    \item (Optional) Generate comprehensive, human-readable malware analysis reports.
\end{enumerate}

\textbf{System Architecture:}
\begin{center}
\includegraphics[width=1.1\textwidth]{architecture_diagram.png}
\end{center}


\textbf{Technology Stack:}
\begin{itemize}[leftmargin=*, itemsep=0pt]
    \item Python 3.10+, PyTorch/Transformers (for LLMs), CAPEv2, MITRE ATT\&CK framework, LaTeX (reporting)
\end{itemize}

\section{Implementation Plan}
\textbf{Timeline and Milestones:}
\begin{itemize}[leftmargin=*, itemsep=0pt]
    \item \textbf{Feb 1-10:} Dataset acquisition and preprocessing
    \item \textbf{Feb 11-20:} Log parser and feature extraction
    \item \textbf{Feb 21-Mar 1:} LLM-based behavior analysis
    \item \textbf{Mar 2-10:} Signature generation and validation
    \item \textbf{Mar 11-15:} MITRE ATT\&CK mapping
    \item \textbf{Mar 16-20:} Report generation, evaluation, and final write-up
    \item \textbf{Mar 22:} Final submission
\end{itemize}

\textbf{Dependencies and Risks:}
\begin{itemize}[leftmargin=*, itemsep=0pt]
    \item LLM access (API limits, cost)
    \item Dataset size/quality
    \item CAPEv2 compatibility
    \item Mitigation: Early testing, fallback to manual feature extraction if needed
\end{itemize}

\section{Research Component}
\subsection{Research Questions}
\begin{enumerate}[leftmargin=*, itemsep=0pt]
    \item What is the precision, recall, and F1-score of LLM-generated CAPEv2 behavioral signatures compared to manually crafted signatures on a held-out malware test set?
    \item To what extent do LLM-generated signatures detect unseen variants from new malware families (i.e., generalization rate on out-of-family samples)?
    \item How accurate is the automated mapping of behaviors to MITRE ATT\&CK techniques (measured by correct technique assignments vs. expert annotation)?
\end{enumerate}


\subsection{Related Work}

\begin{table}[h!]
\centering
\begin{tabular}{|p{2.5cm}|p{3cm}|p{3cm}|p{3cm}|}
\hline
\textbf{Paper/Tool} & \textbf{Approach} & \textbf{Key Contribution} & \textbf{Our Difference} \\
\hline
\href{https://doi.org/10.13140/RG.2.2.23054.16960}{Graph-Ensemble Methods (2024)} & Ensemble of GNNs and NLP for API call sequence analysis & Improves detection accuracy using graph models, automates signature generation with GNNExplainer, maps to MITRE ATT\&CK & We use LLMs for CAPEv2 Python signature generation, not GNNs. \\
\hline
\href{https://arxiv.org/pdf/1711.08336}{DeepSign (2017)} & Deep learning (DBN, denoising autoencoders) for invariant malware signature generation & Achieves 98.6\% accuracy on new malware variants; agnostic to input type (API calls, registry, etc.) & We automate CAPEv2 Python signatures and ATT\&CK mapping with LLMs. \\
\hline
\href{https://github.com/cmikke97/Automatic-Malware-Signature-Generation}{Automatic-Malware-Signature-Generation (GitHub)} & Open-source tool for automated malware signature generation & Provides practical implementation and code for signature automation & We target CAPEv2 format and add ATT\&CK mapping using LLMs. \\
\hline
\end{tabular}
\caption{Related work comparison}
\end{table}

\textbf{Implementations for Comparison:}
\begin{itemize}[leftmargin=*, itemsep=0pt]
    \item \textbf{Graph-Ensemble Methods:} No public code found; method described in detail for reproducibility.
    \item \textbf{DeepSign:} No official code, but method is reproducible from paper.
    \item \textbf{Automatic-Malware-Signature-Generation:} \href{https://github.com/cmikke97/Automatic-Malware-Signature-Generation}{GitHub repository available}.
\end{itemize}

Recent work applies deep learning and graph neural networks to automate malware signature generation and analysis. DeepSign and graph-ensemble methods focus on robust, invariant signature creation from behavioral logs, but do not target CAPEv2 or automate Python signature generation. Open-source tools provide automation but lack ATT\&CK mapping and LLM integration. Our work uniquely automates CAPEv2 Python signature generation, ATT\&CK mapping, and LLM-driven report synthesis for dynamic malware analysis.

\subsection{Dataset Selection}
\begin{itemize}[leftmargin=*, itemsep=0pt]
    \item \textbf{Dataset:} AVAST-CTU CAPEv2 Dataset (\url{https://github.com/avast/avast-ctu-cape-dataset})
    \item \textbf{Justification:} Large, diverse, public, includes CAPEv2 logs
    \item \textbf{Statistics:} $>$100,000 samples, multiple malware families
    \item \textbf{Preprocessing:} Filter for relevant behaviors, split into train/val/test
    \item \textbf{Split Strategy:} Chronological partitioning to avoid data leakage
\end{itemize}

\section{Evaluation Plan}
\begin{itemize}[leftmargin=*, itemsep=0pt]
    \item \textbf{Metrics:} Precision, recall, F1-score (signature quality), coverage, ATT\&CK mapping accuracy, efficiency (time savings)
    \item \textbf{Baselines:} Existing CAPEv2 signatures, manual analyst signatures, rule-based detection
    \item \textbf{Experimental Setup:} Evaluate on held-out test set, compare to baselines, statistical significance via paired t-test
    \item \textbf{Expected Outcomes:} LLM-generated signatures are competitive with manual, improve efficiency, and provide robust ATT\&CK mapping
\end{itemize}

\end{document}
